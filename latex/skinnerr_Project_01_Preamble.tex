%%
%% PACKAGES
%%

% Margins
\usepackage[margin=0.9in, top=0.8in, bottom=1.0in]{geometry}

% Fonts, typesetting, and math symbols
\usepackage[T1]{fontenc}
\usepackage{tgpagella} % Palatino-based font from TeX Gyre
\usepackage[scaled]{beramono} % Lovely monospace font
\usepackage{tgheros} % Helvetica-based font for headings
\usepackage{amsmath, amssymb}
\usepackage{mathtools}
\usepackage{mathdots}
\usepackage{microtype}
\usepackage{xspace}
\usepackage{xfrac}
\usepackage{calc}

% Plotting and drawing
%\usepackage{tikz} % This automatically loads graphicx!
%\usetikzlibrary{calc} % For relative positions to defined coords
%\usepackage{pgfplots} % Scientific plotting tools
%\pgfplotsset{compat=1.7}

% Graphics
\usepackage{graphicx}
%\usepackage[update,prepend]{epstopdf}
\usepackage{titlesec}
\usepackage{color}

% Table improvements
\usepackage{booktabs}

% Figure placement
\usepackage{wrapfig}

% Code listings
\usepackage{listings}
\usepackage{matlab-prettifier}	% MATLAB code listings

% Tweaks for captions and enumerations
\usepackage[labelfont=bf]{caption}
\captionsetup[wrapfigure]{margin=0.5cm}
\usepackage{enumitem}
\setlist[itemize]{itemsep=3pt,leftmargin=*,label=\textbullet}

% Fancy headers
\usepackage{fancyhdr}
\setlength{\headheight}{0pt}
\setlength{\footskip}{50pt}
\renewcommand{\headrulewidth}{0pt}
\renewcommand{\footrulewidth}{0pt}

%%
%% SETTINGS
%%

% Path to look for graphics
\graphicspath{{../images/}}

% Caption spacing
\setlength{\abovecaptionskip}{0pt}

% List spacing
\setlist{noitemsep}

% Table spacing
\renewcommand{\arraystretch}{1.3}

% Math operator font
% Note that cmr=roman and cmss=sans-serif.
\DeclareSymbolFont{sfoperators}{OT1}{cmr}{m}{n}
\DeclareSymbolFontAlphabet{\mathsf}{sfoperators}
\makeatletter
\def\operator@font{\mathgroup\symsfoperators}
\makeatother

%% No indent all paragraphs
%\setlength{\parindent}{0in}

% Figure and table references
\newcommand{\figref}[1]{Figure~\ref{#1}}
\newcommand{\tabref}[1]{Table~\ref{#1}}

% Special format section headings
\titleformat{\section}%
	{\color{blue}\large}% Text formatting
	{Part \arabic{section}}% Number
	{1em}% Space between number and text
	{}% Code before
	[\addvspace{-10pt}\rule{\textwidth}{0.4pt}]% Code after
\titleformat{\subsection}%
	{\color{blue}\normalsize\itshape}% Text formatting
	{Problem \arabic{section}.\arabic{subsection}}% Number
	{1em}% Space between number and text
	{}% Code before
	[\addvspace{-10pt}\rule{\widthof{Problem \arabic{section}.\arabic{subsection}}}{0.4pt}]% Code after
%\titleformat{\subsubsection}%
%	{\color{blue}}% Text formatting
%	{\arabic{subsubsection} $\rightarrow$}% Number
%	{1em}% Space between number and text
%	{}% Code before
%	[]% Code after

\definecolor{mygray}{rgb}{0.4, 0.4, 0.4}
\lstset{
style=Matlab-editor,
mlscaleinline=false,
basicstyle=\ttfamily\lst@ifdisplaystyle\scriptsize\fi,
frame=single,
rulecolor=\color{mygray},
numbers=left,
numbersep=10pt,
numberstyle=\footnotesize \ttfamily \color{mygray},
xleftmargin=30pt,
xrightmargin=5pt,
framexleftmargin=4pt,
framextopmargin=2pt
}

%%
%% COMMANDS
%%

% Superscript text: 1st, 2nd, 3rd, 4th
\newcommand{\suptext}[1]{\ensuremath{^\text{#1}}\xspace}
\newcommand{\st}{\suptext{st}}
\newcommand{\nd}{\suptext{nd}}
\newcommand{\rd}{\suptext{rd}}
\let\oldth\th % Reassign the current \th command
\renewcommand{\th}{\suptext{th}}

% Big O notation
\newcommand{\bigo}{\ensuremath{\mathcal{O}}}

% Bold vectors
\let\oldvec\vec
% Option 1: Works on more than single tokens, but makes regular letters italic as well as bold.
%\renewcommand{\vec}[1]{\mathbold{#1}}
% Option 2: Only works if a single token is passed to the command, but makes regular letters bold only.
\renewcommand{\vec}[1]{
	\ifcat\noexpand#1\relax
		\expandafter\mathbold
	\else
		\expandafter\mathbf
	\fi{{#1}}}

% Underlined vectors and double-underline matrices
\newcommand{\uvec}[1]{\ensuremath{\underline{#1}}}
\newcommand{\umat}[1]{\ensuremath{\underline{\underline{#1}}}}

% Expectation value and mean
\newcommand{\mean}[1]{\ensuremath{\overline{#1}}}
\newcommand{\expectation}[1]{\ensuremath{\left< #1 \right>}}
% Use the following inside text.
\newcommand{\texpectation}[1]{\ensuremath{\langle #1 \rangle}}

% Absolute value and norm bars
\DeclarePairedDelimiter\abs{\lvert}{\rvert}%
\DeclarePairedDelimiter\norm{\lVert}{\rVert}%
% Swap the definition of \abs* and \norm*, so that \abs
% and \norm resizes the size of the brackets, and the 
% starred version does not.
\makeatletter
\let\oldabs\abs
\def\abs{\@ifstar{\oldabs}{\oldabs*}}
\let\oldnorm\norm
\def\norm{\@ifstar{\oldnorm}{\oldnorm*}}
\makeatother

% Vertical asymptote for tikz/pgfplots
\newcommand{\vasymptote}[2][]{
    \draw [densely dashed,#1] ({rel axis cs:0,0} -| {axis cs:#2,0}) -- ({rel axis cs:0,1} -| {axis cs:#2,0});
}

% Vertical dirac delta function for tikz/pgfplots
\newcommand{\diracdelta}[2][]{
    \draw [#1] ({current axis.left of origin} -| {axis cs:#2,0}) -- ({rel axis cs:0,1} -| {axis cs:#2,0});
}